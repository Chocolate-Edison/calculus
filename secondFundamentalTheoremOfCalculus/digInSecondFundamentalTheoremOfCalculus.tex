\documentclass{ximera}

%\usepackage{todonotes}
%\usepackage{mathtools} %% Required for wide table Curl and Greens
%\usepackage{cuted} %% Required for wide table Curl and Greens
\newcommand{\todo}{}

\usepackage{esint} % for \oiint
\ifxake%%https://math.meta.stackexchange.com/questions/9973/how-do-you-render-a-closed-surface-double-integral
\renewcommand{\oiint}{{\large\bigcirc}\kern-1.56em\iint}
\fi


\graphicspath{
  {./}
  {ximeraTutorial/}
  {basicPhilosophy/}
  {functionsOfSeveralVariables/}
  {normalVectors/}
  {lagrangeMultipliers/}
  {vectorFields/}
  {greensTheorem/}
  {shapeOfThingsToCome/}
  {dotProducts/}
  {partialDerivativesAndTheGradientVector/}
  {../ximeraTutorial/}
  {../productAndQuotientRules/exercises/}
  {../motionAndPathsInSpace/exercises/}
  {../normalVectors/exercisesParametricPlots/}
  {../continuityOfFunctionsOfSeveralVariables/exercises/}
  {../partialDerivativesAndTheGradientVector/exercises/}
  {../directionalDerivativeAndChainRule/exercises/}
  {../commonCoordinates/exercisesCylindricalCoordinates/}
  {../commonCoordinates/exercisesSphericalCoordinates/}
  {../greensTheorem/exercisesCurlAndLineIntegrals/}
  {../greensTheorem/exercisesDivergenceAndLineIntegrals/}
  {../shapeOfThingsToCome/exercisesDivergenceTheorem/}
  {../greensTheorem/}
  {../shapeOfThingsToCome/}
  {../separableDifferentialEquations/exercises/}
  {vectorFields/}
}

\newcommand{\mooculus}{\textsf{\textbf{MOOC}\textnormal{\textsf{ULUS}}}}

\usepackage{tkz-euclide}\usepackage{tikz}
\usepackage{tikz-cd}
\usetikzlibrary{arrows}
\tikzset{>=stealth,commutative diagrams/.cd,
  arrow style=tikz,diagrams={>=stealth}} %% cool arrow head
\tikzset{shorten <>/.style={ shorten >=#1, shorten <=#1 } } %% allows shorter vectors

\usetikzlibrary{backgrounds} %% for boxes around graphs
\usetikzlibrary{shapes,positioning}  %% Clouds and stars
\usetikzlibrary{matrix} %% for matrix
\usepgfplotslibrary{polar} %% for polar plots
\usepgfplotslibrary{fillbetween} %% to shade area between curves in TikZ
%\usetkzobj{all} %% obsolete

\usepackage[makeroom]{cancel} %% for strike outs
%\usepackage{mathtools} %% for pretty underbrace % Breaks Ximera
%\usepackage{multicol}
\usepackage{pgffor} %% required for integral for loops

\usepackage{tkz-tab}  %% for sign charts

%% http://tex.stackexchange.com/questions/66490/drawing-a-tikz-arc-specifying-the-center
%% Draws beach ball
\tikzset{pics/carc/.style args={#1:#2:#3}{code={\draw[pic actions] (#1:#3) arc(#1:#2:#3);}}}



\usepackage{array}
\setlength{\extrarowheight}{+.1cm}
\newdimen\digitwidth
\settowidth\digitwidth{9}
\def\divrule#1#2{
\noalign{\moveright#1\digitwidth
\vbox{\hrule width#2\digitwidth}}}





\newcommand{\RR}{\mathbb R}
\newcommand{\R}{\mathbb R}
\newcommand{\N}{\mathbb N}
\newcommand{\Z}{\mathbb Z}

\newcommand{\sagemath}{\textsf{SageMath}}


\renewcommand{\d}{\,d}
%\def\d{\mathop{}\!d}
%\def\d{\,d}
\newcommand{\dd}[2][]{\frac{\d #1}{\d #2}}
\newcommand{\pp}[2][]{\frac{\partial #1}{\partial #2}}
\renewcommand{\l}{\ell}
\newcommand{\ddx}{\frac{d}{\d x}}

\newcommand{\zeroOverZero}{\ensuremath{\boldsymbol{\tfrac{0}{0}}}}
\newcommand{\inftyOverInfty}{\ensuremath{\boldsymbol{\tfrac{\infty}{\infty}}}}
\newcommand{\zeroOverInfty}{\ensuremath{\boldsymbol{\tfrac{0}{\infty}}}}
\newcommand{\zeroTimesInfty}{\ensuremath{\small\boldsymbol{0\cdot \infty}}}
\newcommand{\inftyMinusInfty}{\ensuremath{\small\boldsymbol{\infty - \infty}}}
\newcommand{\oneToInfty}{\ensuremath{\boldsymbol{1^\infty}}}
\newcommand{\zeroToZero}{\ensuremath{\boldsymbol{0^0}}}
\newcommand{\inftyToZero}{\ensuremath{\boldsymbol{\infty^0}}}



\newcommand{\numOverZero}{\ensuremath{\boldsymbol{\tfrac{\#}{0}}}}
\newcommand{\dfn}{\textbf}
%\newcommand{\unit}{\,\mathrm}
\newcommand{\unit}{\mathop{}\!\mathrm}
\newcommand{\eval}[1]{\bigg[ #1 \bigg]}
\newcommand{\seq}[1]{\left( #1 \right)}
\renewcommand{\epsilon}{\varepsilon}
\renewcommand{\phi}{\varphi}


\renewcommand{\iff}{\Leftrightarrow}

\DeclareMathOperator{\arccot}{arccot}
\DeclareMathOperator{\arcsec}{arcsec}
\DeclareMathOperator{\arccsc}{arccsc}
\DeclareMathOperator{\si}{Si}
\DeclareMathOperator{\scal}{scal}
\DeclareMathOperator{\sign}{sign}


%% \newcommand{\tightoverset}[2]{% for arrow vec
%%   \mathop{#2}\limits^{\vbox to -.5ex{\kern-0.75ex\hbox{$#1$}\vss}}}
\newcommand{\arrowvec}[1]{{\overset{\rightharpoonup}{#1}}}
%\renewcommand{\vec}[1]{\arrowvec{\mathbf{#1}}}
\renewcommand{\vec}[1]{{\overset{\boldsymbol{\rightharpoonup}}{\mathbf{#1}}}\hspace{0in}}

\newcommand{\point}[1]{\left(#1\right)} %this allows \vector{ to be changed to \vector{ with a quick find and replace
\newcommand{\pt}[1]{\mathbf{#1}} %this allows \vec{ to be changed to \vec{ with a quick find and replace
\newcommand{\Lim}[2]{\lim_{\point{#1} \to \point{#2}}} %Bart, I changed this to point since I want to use it.  It runs through both of the exercise and exerciseE files in limits section, which is why it was in each document to start with.

\DeclareMathOperator{\proj}{\mathbf{proj}}
\newcommand{\veci}{{\boldsymbol{\hat{\imath}}}}
\newcommand{\vecj}{{\boldsymbol{\hat{\jmath}}}}
\newcommand{\veck}{{\boldsymbol{\hat{k}}}}
\newcommand{\vecl}{\vec{\boldsymbol{\l}}}
\newcommand{\uvec}[1]{\mathbf{\hat{#1}}}
\newcommand{\utan}{\mathbf{\hat{t}}}
\newcommand{\unormal}{\mathbf{\hat{n}}}
\newcommand{\ubinormal}{\mathbf{\hat{b}}}

\newcommand{\dotp}{\bullet}
\newcommand{\cross}{\boldsymbol\times}
\newcommand{\grad}{\boldsymbol\nabla}
\newcommand{\divergence}{\grad\dotp}
\newcommand{\curl}{\grad\cross}
%\DeclareMathOperator{\divergence}{divergence}
%\DeclareMathOperator{\curl}[1]{\grad\cross #1}
\newcommand{\lto}{\mathop{\longrightarrow\,}\limits}

\renewcommand{\bar}{\overline}

\colorlet{textColor}{black}
\colorlet{background}{white}
\colorlet{penColor}{blue!50!black} % Color of a curve in a plot
\colorlet{penColor2}{red!50!black}% Color of a curve in a plot
\colorlet{penColor3}{red!50!blue} % Color of a curve in a plot
\colorlet{penColor4}{green!50!black} % Color of a curve in a plot
\colorlet{penColor5}{orange!80!black} % Color of a curve in a plot
\colorlet{penColor6}{yellow!70!black} % Color of a curve in a plot
\colorlet{fill1}{penColor!20} % Color of fill in a plot
\colorlet{fill2}{penColor2!20} % Color of fill in a plot
\colorlet{fillp}{fill1} % Color of positive area
\colorlet{filln}{penColor2!20} % Color of negative area
\colorlet{fill3}{penColor3!20} % Fill
\colorlet{fill4}{penColor4!20} % Fill
\colorlet{fill5}{penColor5!20} % Fill
\colorlet{gridColor}{gray!50} % Color of grid in a plot

\newcommand{\surfaceColor}{violet}
\newcommand{\surfaceColorTwo}{redyellow}
\newcommand{\sliceColor}{greenyellow}




\pgfmathdeclarefunction{gauss}{2}{% gives gaussian
  \pgfmathparse{1/(#2*sqrt(2*pi))*exp(-((x-#1)^2)/(2*#2^2))}%
}


%%%%%%%%%%%%%
%% Vectors
%%%%%%%%%%%%%

%% Simple horiz vectors
\renewcommand{\vector}[1]{\left\langle #1\right\rangle}


%% %% Complex Horiz Vectors with angle brackets
%% \makeatletter
%% \renewcommand{\vector}[2][ , ]{\left\langle%
%%   \def\nextitem{\def\nextitem{#1}}%
%%   \@for \el:=#2\do{\nextitem\el}\right\rangle%
%% }
%% \makeatother

%% %% Vertical Vectors
%% \def\vector#1{\begin{bmatrix}\vecListA#1,,\end{bmatrix}}
%% \def\vecListA#1,{\if,#1,\else #1\cr \expandafter \vecListA \fi}

%%%%%%%%%%%%%
%% End of vectors
%%%%%%%%%%%%%

%\newcommand{\fullwidth}{}
%\newcommand{\normalwidth}{}



%% makes a snazzy t-chart for evaluating functions
%\newenvironment{tchart}{\rowcolors{2}{}{background!90!textColor}\array}{\endarray}

%%This is to help with formatting on future title pages.
\newenvironment{sectionOutcomes}{}{}



%% Flowchart stuff
%\tikzstyle{startstop} = [rectangle, rounded corners, minimum width=3cm, minimum height=1cm,text centered, draw=black]
%\tikzstyle{question} = [rectangle, minimum width=3cm, minimum height=1cm, text centered, draw=black]
%\tikzstyle{decision} = [trapezium, trapezium left angle=70, trapezium right angle=110, minimum width=3cm, minimum height=1cm, text centered, draw=black]
%\tikzstyle{question} = [rectangle, rounded corners, minimum width=3cm, minimum height=1cm,text centered, draw=black]
%\tikzstyle{process} = [rectangle, minimum width=3cm, minimum height=1cm, text centered, draw=black]
%\tikzstyle{decision} = [trapezium, trapezium left angle=70, trapezium right angle=110, minimum width=3cm, minimum height=1cm, text centered, draw=black]


\outcome{State the Second Fundamental Theorem of Calculus.}
\outcome{Evaluate definite integrals using the Second Fundamental Theorem of Calculus.}
\outcome{Understand how the area under a curve is related to the antiderivative.}
\outcome{Understand the relationship between indefinite and definite integrals.}

\title[Dig-In:]{The Second Fundamental Theorem of Calculus}

\begin{document}
\begin{abstract}
The accumulation of a rate is given by the change in the amount.
\end{abstract}
\maketitle

There is a another common form of the Fundamental Theorem of Calculus. In this form, the Fundamental Theorem of Calculus will
serve as our ``shortcut formula'' for calculating definite integrals.

\begin{theorem}[Second Fundamental Theorem of Calculus]\index{Second Fundamental Theorem of Calculus}
	  Let $f$ be continuous on $[a,b]$. If $F$ is \textbf{any}
	  antiderivative of $f$, then
	  \[
	  \int_a^b f(x)\d x = F(b)-F(a).
	  \]
	  \begin{explanation}
	 	Call $G$ the accumulation function given by
	 	\[G(x) = \int_a^x f(t) \d t.\]
	 	By the First Fundamental Theorem of Calculus, $G$ is an antiderivative of $f$. We see that both:
	 	\begin{align*}
	 		G(a) &= \int_a^a f(t) \d t = 0, \textrm{ and } \\
	 		G(b) &= \int_a^b f(x) \d x.
	 	\end{align*}
	 	Fix $F$ as any antiderivative of $f$. Since $F$ and $G$ have the same derivative, they differ by a constant. 
	 	That is, there is a constant $C$ such that for any $x$ in the interval $[a, b]$, $G(x) = F(x)+C$.
	
		Then:
		\begin{align*}
			\int_a^b f(x) \d x &= G(b) = G(b) - 0\\
				&= G(b)-G(a)\\
				&=( F(b)+C ) - ( F(a) + C )\\
				&= F(b) - F(a).
		\end{align*}
		%    Let $a\le c\le b$ and write
		%    \begin{align*}
		%      \int_a^b f(x) \d x &= \int_a^c f(x) \d x + \int_c^b f(x) \d x \\
		%      &= \int_c^b f(x) \d x - \int_c^a f(x) \d x.
		%    \end{align*}
		%    By the First Fundamental Theorem of Calculus, we have
		%    \[
		%    F(b) = \int_c^b f(x) \d x\qquad\text{and}\qquad F(a) = \int_c^a f(x) \d x
		%    \] 
		%    for some antiderivative $F$ of $f$. So
		%    \[
		%    \int_a^b f(x) \d x = F(b)-F(a)
		%    \]
		%    for this antiderivative. However, \textbf{any} antiderivative
		%    could have be chosen, as antiderivatives of a given function
		%    differ only by a constant, and this constant \textit{always}
		%    cancels out of the expression when evaluating $F(b)-F(a)$.
	\end{explanation}
\end{theorem}

From this you should see that the two versions of the Fundamental Theorem are very closely related. In reality, the two forms are
\textbf{equivalent}, just differently stated. Hence people often simply call them both ``The Fundamental Theorem of Calculus.''
One way of thinking about the Second Fundamental Theorem of Calculus is:
\begin{image}
  \begin{tikzpicture}[scale=2,every node/.style={transform shape}]
    \node at (0,0) {
      $\color{green!70!black!70!blue}\int_a^b\color{blue!70!green}f'(x)\color{green!70!black!70!blue}\d x\color{black} = 
      \color{purple!50!blue!90!black}f(b) - f(a)$
      };
  \end{tikzpicture}
\end{image}
This could be read as:
\begin{quote}\large\textbf{The \textcolor{green!70!black!70!blue}{accumulation} of a \textcolor{blue!70!green}{rate} is given by the \textcolor{purple!50!blue!90!black}{change in the amount}.}

\end{quote}


When we compute a definite integral, we first find an antiderivative and then evaluate at the limits of integration. It is convenient to
first display the antiderivative and then evaluate.  A special notation is often used in the process of evaluating definite integrals
using the Fundamental Theorem of Calculus. Instead of explicitly writing $F(b)-F(a)$, we often write
\[ \eval{F(x)}_a^b \]
meaning that one should evaluate $F(x)$ at $b$ and then subtract $F(x)$ evaluated at $a$
\[ \eval{F(x)}_a^b = F(b)-F(a). \]

\begin{remark}
Let us examine the First and Second Fundamental Theorems together:
\begin{align*}
	\ddx \int_a^x f(t) \d t &= f(x), \textrm{ (First Fundamental Theorem of Calculus)}\\
	\int_a^b \ddx f(x) \d x &= \eval{f(x)}_a^b, \textrm{ (Second Fundamental Theorem of Calculus)}
\end{align*}
Together, they tell us how the definite integral and derivative interact with one-another.
\end{remark}

Let's see some examples of the fundamental theorem in action.

\begin{example}
  Compute:
  \[
  \int_{-2}^2 x^3\d x
  \]
  \begin{explanation}
    We start by finding an antiderivative of $x^3$.  One possible choice is $\frac{x^{4}}{4}$. (We can verify this by differentiating: 
    $\ddx \frac{x^4}{4} = x^3$.) Notice that we do not need the ``$+C$'' term here. The Second Fundamental Theorem of Calculus says
    we can use \textbf{any} antiderivative of $x^3$. We will almost always use the one with $C=0$ when
    evaluating a definite integral.
    \begin{align*}
      \int_{-2}^2 x^3 \d x &=\eval{\frac{x^4}{4}}_{-2}^2\\
      &= \frac{2^4}{4} - \frac{(-2)^4}{4}\\
      &= 0.
    \end{align*}
  \end{explanation}
\end{example}


\begin{example}
	  Compute:
	  \[ \int_0^\pi \sin(\theta) \d \theta \]
	  \begin{explanation}
		    We start by finding an antiderivative of $\sin(\theta)$.  A correct choice is $-\cos(\theta)$, which one could verify this by taking the
		    derivative. Then
		    \begin{align*}
			      \int_0^\pi \sin(\theta) \d \theta &= \eval{-\cos(\theta)}_0^\pi\\
				      &=\answer[given]{-\cos(\pi)} - (-\cos(0))\\
				      &=\answer[given]{2}.
		    \end{align*}
		    This is interesting: It says that the area under one ``hump'' of a sine curve is $2$.
	  \end{explanation}
\end{example}

\begin{example}
  Compute:
  \[
  \int_0^5 e^t\d t
  \]
  \begin{explanation}
    We start by finding an antiderivative of $e^t$.  A correct choice
    is $\answer[given]{e^t}$, one could verify this by taking the
    derivative. Hence
    \begin{align*}
      \int_0^5 e^t \d t &= \eval{e^t}_0^5 \\
      &= e^5-\answer[given]{e^0}\\
      &= e^5-\answer[given]{1}.
    \end{align*}
  \end{explanation}
\end{example}


\begin{example}
	Compute:
	\[ \int_1^2\left(x^9 + \frac{1}{x}\right) \d x \]
	\begin{explanation}
		We start by finding an antiderivative of $x^9 + \frac{1}{x}$.  A
		correct choice is $\frac{x^{10}}{10} + \ln(x)$, one could verify this
		by taking the derivative. Hence
		\begin{align*}
			\int_1^2\left(x^9 + \frac{1}{x}\right) \d x &= \eval{\frac{x^{10}}{10} + \ln(x)}_1^2 \\
				&= \frac{2^{10}}{10} + \ln(2) - \answer[given]{\frac{1}{10}}.
		\end{align*}
	\end{explanation}
\end{example}




\section{Understanding motion with the Fundamental Theorem of Calculus}

We know that
\begin{itemize}
	\item The derivative of a position function is a velocity function.
	\item The derivative of a velocity function is an acceleration  function.
\end{itemize}
Now consider definite integrals of velocity and acceleration
functions. Specifically, if $v(t)$ is a velocity function, what does $\displaystyle \int_a^b v(t)\d t$ mean?

The Second Fundamental Theorem of Calculus states that
\[ \int_a^b v(t)\d t = V(b) - V(a), \]
where $V(t)$ is any antiderivative of $v(t)$. Since $v(t)$ is a
velocity function, we can choose  $V(t)$ to be the position function. Then, $V(b) -
V(a)$ measures a \textbf{change in position}, or \dfn{displacement} over the time interval $[a,b]$.

\begin{example}
  A ball is thrown straight up with velocity given by $v(t) = -32t+20 \unit{ft/s}$, where $t$ is measured in $\unit{seconds}$. Find, and interpret,
  $\displaystyle \int_0^1 v(t)\d t$.
    \begin{explanation}
      Using the Second Fundamental Theorem of Calculus, we have
      \begin{align*}
        	\int_0^1 v(t)\d t &= \int_0^1 (-32t+20)\d t \\
		&= \eval{\answer[given]{-16t^2 + 20t}}_0^1 \\
		&= 4.
      \end{align*}
      Thus if a ball is thrown straight up into the air with velocity
      \[  v(t) = \answer[given]{-32t+20}, \]
      the height of the ball, $1 \unit{second}$ later, will be $4 \unit{feet}$ above the
      initial height. Note that the ball has \textit{traveled} much
      farther. It has gone up to its peak and is falling down, but the
      difference between its height at $t=0$ and $t=1$ is $4 \unit{ft}$. 
    \end{explanation}
\end{example}    



Now we know that to solve certain kinds of problems, those that involve accumulation of some form, we ``merely'' find an
antiderivative and substitute two values and subtract. Unfortunately, finding antiderivatives can be quite difficult. While there are a
small number of rules that allow us to compute the derivative of any common function, there are no such rules for antiderivatives. There
are some techniques that frequently prove useful, but we will never be able to reduce the problem to a completely mechanical process.




\end{document}
