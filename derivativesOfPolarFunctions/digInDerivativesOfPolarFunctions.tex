\documentclass{ximera}

%\usepackage{todonotes}
%\usepackage{mathtools} %% Required for wide table Curl and Greens
%\usepackage{cuted} %% Required for wide table Curl and Greens
\newcommand{\todo}{}

\usepackage{esint} % for \oiint
\ifxake%%https://math.meta.stackexchange.com/questions/9973/how-do-you-render-a-closed-surface-double-integral
\renewcommand{\oiint}{{\large\bigcirc}\kern-1.56em\iint}
\fi


\graphicspath{
  {./}
  {ximeraTutorial/}
  {basicPhilosophy/}
  {functionsOfSeveralVariables/}
  {normalVectors/}
  {lagrangeMultipliers/}
  {vectorFields/}
  {greensTheorem/}
  {shapeOfThingsToCome/}
  {dotProducts/}
  {partialDerivativesAndTheGradientVector/}
  {../ximeraTutorial/}
  {../productAndQuotientRules/exercises/}
  {../motionAndPathsInSpace/exercises/}
  {../normalVectors/exercisesParametricPlots/}
  {../continuityOfFunctionsOfSeveralVariables/exercises/}
  {../partialDerivativesAndTheGradientVector/exercises/}
  {../directionalDerivativeAndChainRule/exercises/}
  {../commonCoordinates/exercisesCylindricalCoordinates/}
  {../commonCoordinates/exercisesSphericalCoordinates/}
  {../greensTheorem/exercisesCurlAndLineIntegrals/}
  {../greensTheorem/exercisesDivergenceAndLineIntegrals/}
  {../shapeOfThingsToCome/exercisesDivergenceTheorem/}
  {../greensTheorem/}
  {../shapeOfThingsToCome/}
  {../separableDifferentialEquations/exercises/}
  {vectorFields/}
}

\newcommand{\mooculus}{\textsf{\textbf{MOOC}\textnormal{\textsf{ULUS}}}}

\usepackage{tkz-euclide}\usepackage{tikz}
\usepackage{tikz-cd}
\usetikzlibrary{arrows}
\tikzset{>=stealth,commutative diagrams/.cd,
  arrow style=tikz,diagrams={>=stealth}} %% cool arrow head
\tikzset{shorten <>/.style={ shorten >=#1, shorten <=#1 } } %% allows shorter vectors

\usetikzlibrary{backgrounds} %% for boxes around graphs
\usetikzlibrary{shapes,positioning}  %% Clouds and stars
\usetikzlibrary{matrix} %% for matrix
\usepgfplotslibrary{polar} %% for polar plots
\usepgfplotslibrary{fillbetween} %% to shade area between curves in TikZ
%\usetkzobj{all} %% obsolete

\usepackage[makeroom]{cancel} %% for strike outs
%\usepackage{mathtools} %% for pretty underbrace % Breaks Ximera
%\usepackage{multicol}
\usepackage{pgffor} %% required for integral for loops

\usepackage{tkz-tab}  %% for sign charts

%% http://tex.stackexchange.com/questions/66490/drawing-a-tikz-arc-specifying-the-center
%% Draws beach ball
\tikzset{pics/carc/.style args={#1:#2:#3}{code={\draw[pic actions] (#1:#3) arc(#1:#2:#3);}}}



\usepackage{array}
\setlength{\extrarowheight}{+.1cm}
\newdimen\digitwidth
\settowidth\digitwidth{9}
\def\divrule#1#2{
\noalign{\moveright#1\digitwidth
\vbox{\hrule width#2\digitwidth}}}





\newcommand{\RR}{\mathbb R}
\newcommand{\R}{\mathbb R}
\newcommand{\N}{\mathbb N}
\newcommand{\Z}{\mathbb Z}

\newcommand{\sagemath}{\textsf{SageMath}}


\renewcommand{\d}{\,d}
%\def\d{\mathop{}\!d}
%\def\d{\,d}
\newcommand{\dd}[2][]{\frac{\d #1}{\d #2}}
\newcommand{\pp}[2][]{\frac{\partial #1}{\partial #2}}
\renewcommand{\l}{\ell}
\newcommand{\ddx}{\frac{d}{\d x}}

\newcommand{\zeroOverZero}{\ensuremath{\boldsymbol{\tfrac{0}{0}}}}
\newcommand{\inftyOverInfty}{\ensuremath{\boldsymbol{\tfrac{\infty}{\infty}}}}
\newcommand{\zeroOverInfty}{\ensuremath{\boldsymbol{\tfrac{0}{\infty}}}}
\newcommand{\zeroTimesInfty}{\ensuremath{\small\boldsymbol{0\cdot \infty}}}
\newcommand{\inftyMinusInfty}{\ensuremath{\small\boldsymbol{\infty - \infty}}}
\newcommand{\oneToInfty}{\ensuremath{\boldsymbol{1^\infty}}}
\newcommand{\zeroToZero}{\ensuremath{\boldsymbol{0^0}}}
\newcommand{\inftyToZero}{\ensuremath{\boldsymbol{\infty^0}}}



\newcommand{\numOverZero}{\ensuremath{\boldsymbol{\tfrac{\#}{0}}}}
\newcommand{\dfn}{\textbf}
%\newcommand{\unit}{\,\mathrm}
\newcommand{\unit}{\mathop{}\!\mathrm}
\newcommand{\eval}[1]{\bigg[ #1 \bigg]}
\newcommand{\seq}[1]{\left( #1 \right)}
\renewcommand{\epsilon}{\varepsilon}
\renewcommand{\phi}{\varphi}


\renewcommand{\iff}{\Leftrightarrow}

\DeclareMathOperator{\arccot}{arccot}
\DeclareMathOperator{\arcsec}{arcsec}
\DeclareMathOperator{\arccsc}{arccsc}
\DeclareMathOperator{\si}{Si}
\DeclareMathOperator{\scal}{scal}
\DeclareMathOperator{\sign}{sign}


%% \newcommand{\tightoverset}[2]{% for arrow vec
%%   \mathop{#2}\limits^{\vbox to -.5ex{\kern-0.75ex\hbox{$#1$}\vss}}}
\newcommand{\arrowvec}[1]{{\overset{\rightharpoonup}{#1}}}
%\renewcommand{\vec}[1]{\arrowvec{\mathbf{#1}}}
\renewcommand{\vec}[1]{{\overset{\boldsymbol{\rightharpoonup}}{\mathbf{#1}}}\hspace{0in}}

\newcommand{\point}[1]{\left(#1\right)} %this allows \vector{ to be changed to \vector{ with a quick find and replace
\newcommand{\pt}[1]{\mathbf{#1}} %this allows \vec{ to be changed to \vec{ with a quick find and replace
\newcommand{\Lim}[2]{\lim_{\point{#1} \to \point{#2}}} %Bart, I changed this to point since I want to use it.  It runs through both of the exercise and exerciseE files in limits section, which is why it was in each document to start with.

\DeclareMathOperator{\proj}{\mathbf{proj}}
\newcommand{\veci}{{\boldsymbol{\hat{\imath}}}}
\newcommand{\vecj}{{\boldsymbol{\hat{\jmath}}}}
\newcommand{\veck}{{\boldsymbol{\hat{k}}}}
\newcommand{\vecl}{\vec{\boldsymbol{\l}}}
\newcommand{\uvec}[1]{\mathbf{\hat{#1}}}
\newcommand{\utan}{\mathbf{\hat{t}}}
\newcommand{\unormal}{\mathbf{\hat{n}}}
\newcommand{\ubinormal}{\mathbf{\hat{b}}}

\newcommand{\dotp}{\bullet}
\newcommand{\cross}{\boldsymbol\times}
\newcommand{\grad}{\boldsymbol\nabla}
\newcommand{\divergence}{\grad\dotp}
\newcommand{\curl}{\grad\cross}
%\DeclareMathOperator{\divergence}{divergence}
%\DeclareMathOperator{\curl}[1]{\grad\cross #1}
\newcommand{\lto}{\mathop{\longrightarrow\,}\limits}

\renewcommand{\bar}{\overline}

\colorlet{textColor}{black}
\colorlet{background}{white}
\colorlet{penColor}{blue!50!black} % Color of a curve in a plot
\colorlet{penColor2}{red!50!black}% Color of a curve in a plot
\colorlet{penColor3}{red!50!blue} % Color of a curve in a plot
\colorlet{penColor4}{green!50!black} % Color of a curve in a plot
\colorlet{penColor5}{orange!80!black} % Color of a curve in a plot
\colorlet{penColor6}{yellow!70!black} % Color of a curve in a plot
\colorlet{fill1}{penColor!20} % Color of fill in a plot
\colorlet{fill2}{penColor2!20} % Color of fill in a plot
\colorlet{fillp}{fill1} % Color of positive area
\colorlet{filln}{penColor2!20} % Color of negative area
\colorlet{fill3}{penColor3!20} % Fill
\colorlet{fill4}{penColor4!20} % Fill
\colorlet{fill5}{penColor5!20} % Fill
\colorlet{gridColor}{gray!50} % Color of grid in a plot

\newcommand{\surfaceColor}{violet}
\newcommand{\surfaceColorTwo}{redyellow}
\newcommand{\sliceColor}{greenyellow}




\pgfmathdeclarefunction{gauss}{2}{% gives gaussian
  \pgfmathparse{1/(#2*sqrt(2*pi))*exp(-((x-#1)^2)/(2*#2^2))}%
}


%%%%%%%%%%%%%
%% Vectors
%%%%%%%%%%%%%

%% Simple horiz vectors
\renewcommand{\vector}[1]{\left\langle #1\right\rangle}


%% %% Complex Horiz Vectors with angle brackets
%% \makeatletter
%% \renewcommand{\vector}[2][ , ]{\left\langle%
%%   \def\nextitem{\def\nextitem{#1}}%
%%   \@for \el:=#2\do{\nextitem\el}\right\rangle%
%% }
%% \makeatother

%% %% Vertical Vectors
%% \def\vector#1{\begin{bmatrix}\vecListA#1,,\end{bmatrix}}
%% \def\vecListA#1,{\if,#1,\else #1\cr \expandafter \vecListA \fi}

%%%%%%%%%%%%%
%% End of vectors
%%%%%%%%%%%%%

%\newcommand{\fullwidth}{}
%\newcommand{\normalwidth}{}



%% makes a snazzy t-chart for evaluating functions
%\newenvironment{tchart}{\rowcolors{2}{}{background!90!textColor}\array}{\endarray}

%%This is to help with formatting on future title pages.
\newenvironment{sectionOutcomes}{}{}



%% Flowchart stuff
%\tikzstyle{startstop} = [rectangle, rounded corners, minimum width=3cm, minimum height=1cm,text centered, draw=black]
%\tikzstyle{question} = [rectangle, minimum width=3cm, minimum height=1cm, text centered, draw=black]
%\tikzstyle{decision} = [trapezium, trapezium left angle=70, trapezium right angle=110, minimum width=3cm, minimum height=1cm, text centered, draw=black]
%\tikzstyle{question} = [rectangle, rounded corners, minimum width=3cm, minimum height=1cm,text centered, draw=black]
%\tikzstyle{process} = [rectangle, minimum width=3cm, minimum height=1cm, text centered, draw=black]
%\tikzstyle{decision} = [trapezium, trapezium left angle=70, trapezium right angle=110, minimum width=3cm, minimum height=1cm, text centered, draw=black]


\outcome{Compute $\frac{dy}{dx}$ from the polar representation of a curve.}
\outcome{Determine where $\frac{dy}{dx}$ of a polar curve is zero or undefined.}
\outcome{Find the equation of a tangent line to the polar representation of a curve.}

\title[Dig-In:]{Derivatives of polar functions}

\begin{document}
\begin{abstract}
  We differentiate polar functions.
\end{abstract}
\maketitle

The previous section discussed a special class of parametric functions
called polar functions.  We know that
\begin{align*}
  \d x &= x'(t) \d t\\
  \d y &= y'(t) \d t,
\end{align*}
and so we can compute the derivative of $y$ with respect to $x$ using
differentials:
\[
\frac{\d y}{\d x} = \frac{y'(t) \d t}{x'(t) \d t} = \frac{y'(t)}{x'(t)}
\]
provided that $x'(t) \ne 0$.  With polar functions we have
\begin{align*}
  x(\theta) &= r(\theta) \cdot \cos(\theta) \\
  y(\theta) &= r(\theta) \cdot \sin(\theta),
\end{align*}
so
\begin{align*}
\dd[y]{x} &= \frac{y'(\theta)}{x'(\theta)} \\
&= \frac{r'(\theta) \sin(\theta)+r(\theta) \cos(\theta)}{r'(\theta)\cos(\theta)-r(\theta)\sin(\theta)}
\end{align*}
provided that $x'(\theta)\ne 0$.

\begin{example}
  Consider the lima\c{c}on $r(\theta) =1+2\sin(\theta)$ on the interval $[0,2\pi]$:
  \begin{image}
     \begin{tikzpicture}
          \begin{polaraxis}[
              xtick={0,45,...,360},
              xticklabels={$0$,$\frac{\pi}{4}$,$\frac{\pi}{2}$,$\frac{3\pi}{4}$,$\pi$,$\frac{5\pi}{4}$,$\frac{3\pi}{2}$,$\frac{7\pi}{4}$,$2\pi$},
            ]
            \addplot+[very thick, mark=none,penColor,domain=0:360,samples=100,smooth] {1+2*sin(x)};
          \end{polaraxis}
     \end{tikzpicture}
  \end{image}
  Find the equation of the tangent line to the curve at $\theta=\pi/4$.
  \begin{explanation}
    We start by computing the derivative in rectangular coordinates. Recall that
    \begin{align*}
      x(\theta) &= \left(1+2\sin(\theta)\right)\cdot \cos(\theta) = \cos(\theta)+2\sin(\theta)\cos(\theta)\\
      y(\theta) &= \left(1+2\sin(\theta)\right)\cdot \sin(\theta) = \sin(\theta)+2\sin^2(\theta)
    \end{align*}
    Consequently,
    \begin{align*}
      x'(\theta) &= \answer[given]{-\sin(\theta) + 2\cos^2(\theta)-2\sin^2(\theta)} \\
      y'(\theta) &= \answer[given]{\cos(\theta) + 4\sin(\theta)\cos(\theta)}
    \end{align*}
    and therefore
    \[
    \dd[y]{x} = \answer[given]{\frac{\cos(\theta) + 4\sin(\theta)\cos(\theta)}{-\sin(\theta) + 2\cos^2(\theta)-2\sin^2(\theta)}}.
    \]
    When $\theta=\pi/4$,
    \begin{align*}
      x(\theta) &= \answer[given]{1+\sqrt{2}/2}\\
      y(\theta) &=\answer[given]{1+\sqrt{2}/2}\\
    \dd[y]{x} &=\answer[given]{-2\sqrt{2}-1}
    \end{align*}
     Thus the equation of the line (in \wordChoice{\choice{polar}\choice[correct]{rectangular}} coordinates) tangent to the lima\c{c}on at $\theta=\pi/4$ is
     \[
     y=(-2\sqrt{2}-1)\big(x-(1+\sqrt{2}/2)\big)+1+\sqrt{2}/2 \approx  -3.83 x+8.24.
     \]
    \begin{prompt}
      \[
      \graph{r=1+2\sin(\theta), y=(-2\sqrt{2}-1)(x-(1+\sqrt{2}/2))+1+\sqrt{2}/2}
      \]
    \end{prompt}
  \end{explanation}
\end{example}





\begin{example}
  Consider the lima\c{c}on given by $r(\theta) =1+2\sin(\theta)$ on the
  interval $[0,2\pi]$.  Find Cartesian coordinates for the points
  where the tangent line to this lima\c{c}on is horizontal.
\begin{explanation}
  To find these points where the tangent lines are horizontal, we seek
  points where $\dd[y]{x} = \answer[given]{0}$.  From earlier, we discovered
  \[
  \dd[y]{x} =\frac{\cos(\theta) + 4\sin(\theta)\cos(\theta)}{-\sin(\theta) + 2\cos^2(\theta)-2\sin^2(\theta)}
  \]
  so now we must find where the numerator is $0$. Since the numerator is equal to
  \[
  \cos(\theta)(1+ 4\sin(\theta))
  \]
  and since a product is zero when \wordChoice{\choice{both}\choice[correct]{either}} of the terms is zero,
  the numerator is zero when 
  \begin{itemize}
  \item $\cos(\theta)=0$ or
  \item $1+4\sin(\theta)=0$.
  \end{itemize}

  Let's examine the case of $\cos(\theta) = 0$ first, and restrict to the situation where $\theta$ is between $0$ and $2\pi$.
  If $\theta$ is in the interval $[0,\pi]$, then $\cos(\theta)=0$ when $\theta=\answer[given]{\pi/2}$.
  If $\theta$ is in the interval $[\pi,2\pi]$, then $\cos(\theta)=0$ when $\theta=\answer[given]{3\pi/2}$.

  The corresponding points in rectangular coordinates are
  \begin{align*}
    x(\pi/2) &= \left(1+2\sin(\pi/2)\right)\cdot\cos(\pi/2)\\
    &= 0,\\
    y(\pi/2) &= \left(1+2\sin(\pi/2)\right)\cdot\sin(\pi/2)\\
    &= 3,
  \end{align*}
  meaning $(x,y) = (0,3)$, and
  \begin{align*}
    x(3\pi/2) &= \left(1+2\sin(3\pi/2)\right)\cdot\cos(3\pi/2)\\
    &= 0,\\
    y(3\pi/2) &= \left(1+2\sin(3\pi/2)\right)\cdot\sin(3\pi/2)\\
    &= 1,
  \end{align*}
  meaning $(x,y) = (0,1)$.

  But the numerator was the product of $\cos(\theta)$ and another term, namely $1+4\sin(\theta)$  So the numerator is also zero when
  \begin{align*}
    1+4\sin(\theta)&=0\\ 4\sin(\theta)&=-1\\ \sin(\theta)&=\frac{-1}{4}.
  \end{align*}
  At this point we have a choice to make: We can either apply the
  arcsin function and solve for $\theta$, or we can work with
  $\sin(\theta)$ directly. Perhaps the easier route is to work with
  $\sin(\theta)$ directly. To explore $\sin(\theta)$, we may choose an
  algebraic method employing the Pythagorean identity, or a geometric
  method looking at the unit circle with the Pythagorean
  theorem. Let's take the geometric route to compute $\cos(\theta)$.
  \begin{image}
    \begin{tikzpicture}
      \begin{axis}[
          xmin=-1.1,xmax=1.1,ymin=-1.1,ymax=1.1,
          axis lines=center,
          width=4in,
          ticks=none,
          clip=false,
          unit vector ratio*=1 1 1,
          %xlabel=$x$, ylabel=$y$,
          every axis y label/.style={at=(current axis.above origin),anchor=south},
          every axis x label/.style={at=(current axis.right of origin),anchor=west},
        ]        
        \addplot [black!50!white, dashed, smooth, domain=(0:360)] ({cos(x)},{sin(x)}); %% unit circle
        
        \addplot [ultra thick] plot coordinates {(0,0) (.97,-.25)}; 
        \addplot [ultra thick] plot coordinates {(.97,0) (.97,-.25)}; 
        \addplot [ultra thick] plot coordinates {(0,0) (.97,0)};

        \addplot [ultra thick] plot coordinates {(0,0) (-.97,-.25)}; 
        \addplot [ultra thick] plot coordinates {(-.97,0) (-.97,-.25)}; 
        \addplot [ultra thick] plot coordinates {(0,0) (-.97,0)}; 

        \node at (axis cs:-.5,-.2) {$1$};
        \node at (axis cs:.5,-.2) {$1$};
        \node[anchor=west] at (axis cs:1,-.11) {$\sin(\theta) = -1/4$};
        \node[anchor=east] at (axis cs:-1,-.11) {$\sin(\theta) = -1/4$};
        \node[anchor=south] at (axis cs:-.5,0) {$\cos(\theta) = -\sqrt{15}/4$};
        \node[anchor=south] at (axis cs:.5,0) {$\cos(\theta) = \sqrt{15}/4$};
        
      \end{axis}
    \end{tikzpicture}
  \end{image}
  Hence the derivative \wordChoice{\choice[correct]{is zero}\choice{is one}\choice{is undefined}} at
  \begin{align*}
    x(\theta) &= \left(1+2\sin(\theta)\right)\cdot\cos(\theta)\\
    &= \left(1+2(-1/4)\right)\cdot \sqrt{15}/4\\
    &= \sqrt{15}/8,\\
    y(\theta) &= \left(1+2\sin(\theta)\right)\cdot\sin(\theta)\\
    &= \left(1+2(-1/4)\right)\cdot (-1/4)\\
    &= -1/8,
  \end{align*}
  meaning $(x,y) = (\sqrt{15}/8, -1/8)$, and
   \begin{align*}
    x(\theta) &= \left(1+2\sin(\theta)\right)\cdot\cos(\theta)\\
    &= \left(1+2(-1/4)\right)\cdot (-\sqrt{15}/4)\\
    &= -\sqrt{15}/8,\\
    y(\theta) &= \left(1+2\sin(\theta)\right)\cdot\sin(\theta)\\
    &= \left(1+2(-1/4)\right)\cdot (-1/4)\\
    &= -1/8,
  \end{align*}
   meaning $(x,y) = (-\sqrt{15}/8, -1/8)$.

   In summary, the graph of $r(\theta) = 1+2\sin(\theta)$ on the interval $[0,2\pi]$
   has horizontal tangent lines at the points
   \begin{itemize}
   \item $(x,y) = (0,\answer[given]{3})$,
   \item $(x,y) = (0,1)$,
   \item $(x,y) = (\sqrt{15}/8, -1/8)$, and 
   \item $(x,y) = (-\sqrt{15}/8, -1/8)$.
   \end{itemize}
   \begin{prompt}
     You can confirm this by looking at the graph below:
     \[
     \graph{r=1+2\sin(\theta),y=3,y=1,y=-1/8}
     \]
   \end{prompt}
\end{explanation}
\end{example}


\begin{example}
  Again consider our friend the lima\c{c}on given by $r(\theta)
  =1+2\sin(\theta)$ on the interval $[0,2\pi]$.  Find Cartesian
  coordinates for the points where the tangent line to this lima\c{c}on
  is vertical.
  \begin{explanation}
    To find the vertical tangent lines, we seek points where the
    denominator of
    \[
    \dd[y]{x}=\frac{\cos(\theta) + 4\sin(\theta)\cos(\theta)}{-\sin(\theta) + 2\cos^2(\theta)-2\sin^2(\theta)}
    \]
    is equal to zero.  To start, we will convert $\cos^2(\theta)$ to
    $1-\answer[given]{\sin^2(\theta)}$, and express the denominator as a quadratic
    expression in $\sin(\theta)$:
  \begin{align*}
    -\sin(\theta) + 2\cos^2(\theta)-2\sin^2(\theta) & = -\sin(\theta) + 2(1-\sin^2(\theta))-2\sin^2(\theta) \\
    &= \answer[given]{-4} \sin^2(\theta) + \answer[given]{-1}\sin(\theta) + \answer[given]{2}.
  \end{align*}
  We then set this equal to zero and note that the equation is quadratic equation in the variable $\sin(\theta)$. The quadratic formula gives that 
  \begin{align*}
    \sin(\theta) &= \frac{-1\pm\sqrt{\answer[given]{33}}}{8}.
  \end{align*}
  Now we can find cosine either geometrically via the unit circle or algebraically with the Pythagorean
  identity.  This time, let's take the more algebraic route with the Pythagorean identity. When $\sin(\theta) =\frac{-1+\sqrt{33}}{8}$, then
  \begin{align*}
    \cos^2(\theta) + \sin^2(\theta) &= 1\\
    \cos^2(\theta) + \left(\frac{-1+\sqrt{33}}{8}\right) &= 1.
  \end{align*}
  So
  \[
  \cos(\theta) = \pm \frac{\sqrt{(\answer[given]{15}+\sqrt{33})/2}}{4}
  \]
  and when $\sin(\theta) =\frac{-1-\sqrt{33}}{8}$,
  \begin{align*}
    \cos^2(\theta) + \sin^2(\theta) &= 1\\
    \cos^2(\theta) + \left(\frac{-1-\sqrt{33}}{8}\right) &= 1.
  \end{align*}
  So
  \[
  \cos(\theta) = \pm \frac{\sqrt{(\answer[given]{15}-\sqrt{33})/2}}{4}.
  \]
  So now we know that the graph of $r(\theta) =1+2\sin(\theta)$ on
  $[0,2\pi]$ has vertical tangent lines at the four points
   \begin{image}
      \begin{tikzpicture}
        \node at (0,0) {
          $\begin{aligned}
          (\cos(\theta),\sin(\theta)) = \left(\frac{\sqrt{(15+\sqrt{33})/2}}{4},\frac{-1+\sqrt{33}}{8}\right),\\
          (\cos(\theta),\sin(\theta)) = \left(\frac{-\sqrt{(15+\sqrt{33})/2}}{4},\frac{-1+\sqrt{33}}{8}\right),\\
          (\cos(\theta),\sin(\theta)) = \left(\frac{\sqrt{(15-\sqrt{33})/2}}{4},\frac{-1-\sqrt{33}}{8}\right), \\ 
          (\cos(\theta),\sin(\theta)) = \left(\frac{-\sqrt{(15-\sqrt{33})/2}}{4},\frac{-1-\sqrt{33}}{8}\right).
          \end{aligned}$
        };
      \end{tikzpicture}
   \end{image}
   Recalling that
   \begin{align*}
     x(\theta) &= \left(1+2\sin(\theta)\right)\cdot\cos(\theta)\\
     y(\theta) &= \left(1+2\sin(\theta)\right)\cdot\sin(\theta)
   \end{align*}
   we see that the graph of $r(\theta) =1+2\sin(\theta)$ on $[0,2\pi]$
   has vertical tangent lines at
    \begin{image}
      \begin{tikzpicture}
        \node at (0,0) {
          $\begin{aligned}
            (x,y) &= \left(\left(1+\frac{-1+\sqrt{33}}{4}\right)\frac{\sqrt{(15+\sqrt{33})/2}}{4},\left(1+\frac{-1+\sqrt{33}}{4}\right)\frac{-1+\sqrt{33}}{8}\right),\\
            (x,y) &= \left(\left(1+\frac{-1+\sqrt{33}}{4}\right)\frac{-\sqrt{(15+\sqrt{33})/2}}{4},\left(1+\frac{-1+\sqrt{33}}{4}\right)\frac{-1+\sqrt{33}}{8}\right),\\
            (x,y) &= \left(\left(1+\frac{-1-\sqrt{33}}{4}\right)\frac{\sqrt{(15-\sqrt{33})/2}}{4},\left(1+\frac{-1-\sqrt{33}}{4}\right)\frac{-1-\sqrt{33}}{8}\right),\\
            (x,y) &= \left(\left(1+\frac{-1-\sqrt{33}}{4}\right)\frac{-\sqrt{(15-\sqrt{33})/2}}{4},\left(1+\frac{-1-\sqrt{33}}{4}\right)\frac{-1-\sqrt{33}}{8}\right).
          \end{aligned}$
        };
      \end{tikzpicture}
    \end{image}
   
   \begin{prompt}
     You can confirm this visually by looking at the graph below:
     \[
     \graph{r=1+2\sin(\theta),
       x=\left(1+\frac{-1+\sqrt{33}}{4}\right)\frac{\sqrt{(15+\sqrt{33})/2}}{4},
       x=\left(1+\frac{-1+\sqrt{33}}{4}\right)\frac{-\sqrt{(15+\sqrt{33})/2}}{4},
       x=\left(1+\frac{-1-\sqrt{33}}{4}\right)\frac{\sqrt{(15-\sqrt{33})/2}}{4},
       x=\left(1+\frac{-1-\sqrt{33}}{4}\right)\frac{-\sqrt{(15-\sqrt{33})/2}}{4}}
     \]
   \end{prompt}
  \end{explanation}
\end{example}

When the graph of the polar function $r(\theta)$ intersects the origin
(sometimes called the ``pole''), then $r(\alpha)=0$ for some angle
$\alpha$.

\begin{question}
  When $r(\alpha) = 0$, what is the formula for $\dd[y]{x}$?
\begin{multipleChoice} 
  \choice{$\dd[y]{x}= \sin(\alpha)$.}
  \choice{$\dd[y]{x}= \cos(\alpha)$.}  
  \choice[correct]{$\dd[y]{x}= \tan(\alpha)$.}
\end{multipleChoice}
\end{question}
The answer to the question above leads us to an interesting point. It
tells us the slope of the tangent line at the pole.  When a polar
graph touches the pole at $\theta=\alpha$, the equation of the tangent
line in polar coordinates at the pole is $\theta=\alpha$.


\begin{example}
  Find the equations of the lines tangent to the graph of
  $r(\theta)=1+2\sin(\theta)$ at the pole.
  \begin{explanation}
    We need to know when $r(\theta)=0$.
    \begin{align*}
      1+2\sin(\theta) &= 0\\
      \sin(\theta) &= -1/2\\
      \theta &= \frac{7\pi}{6},\ \frac{\answer[given]{11}\pi}6.
    \end{align*}
    Thus the equations of the tangent lines, in \wordChoice{\choice[correct]{polar}\choice{rectangular}} coordinates, are
    \begin{align*}
      \theta &=7\pi/6 \theta \\
      \theta &= \answer[given]{11}\pi/6.
    \end{align*}
    In rectangular form, the tangent lines are $y=\tan(7\pi/6)x$ and
    $y=\answer[given]{\tan(11\pi/6)}x$.
    \begin{prompt}
     You can confirm this by looking at the graph below:
     \[
     \graph{r=1+2\sin(\theta),y=\tan(7\pi/6)x,y=\tan(11\pi/6)x}
     \]
   \end{prompt}
  \end{explanation}
\end{example}
\end{document}
















