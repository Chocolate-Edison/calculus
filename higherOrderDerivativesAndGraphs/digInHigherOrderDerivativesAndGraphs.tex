\documentclass{ximera}

%\usepackage{todonotes}
%\usepackage{mathtools} %% Required for wide table Curl and Greens
%\usepackage{cuted} %% Required for wide table Curl and Greens
\newcommand{\todo}{}

\usepackage{esint} % for \oiint
\ifxake%%https://math.meta.stackexchange.com/questions/9973/how-do-you-render-a-closed-surface-double-integral
\renewcommand{\oiint}{{\large\bigcirc}\kern-1.56em\iint}
\fi


\graphicspath{
  {./}
  {ximeraTutorial/}
  {basicPhilosophy/}
  {functionsOfSeveralVariables/}
  {normalVectors/}
  {lagrangeMultipliers/}
  {vectorFields/}
  {greensTheorem/}
  {shapeOfThingsToCome/}
  {dotProducts/}
  {partialDerivativesAndTheGradientVector/}
  {../ximeraTutorial/}
  {../productAndQuotientRules/exercises/}
  {../motionAndPathsInSpace/exercises/}
  {../normalVectors/exercisesParametricPlots/}
  {../continuityOfFunctionsOfSeveralVariables/exercises/}
  {../partialDerivativesAndTheGradientVector/exercises/}
  {../directionalDerivativeAndChainRule/exercises/}
  {../commonCoordinates/exercisesCylindricalCoordinates/}
  {../commonCoordinates/exercisesSphericalCoordinates/}
  {../greensTheorem/exercisesCurlAndLineIntegrals/}
  {../greensTheorem/exercisesDivergenceAndLineIntegrals/}
  {../shapeOfThingsToCome/exercisesDivergenceTheorem/}
  {../greensTheorem/}
  {../shapeOfThingsToCome/}
  {../separableDifferentialEquations/exercises/}
  {vectorFields/}
}

\newcommand{\mooculus}{\textsf{\textbf{MOOC}\textnormal{\textsf{ULUS}}}}

\usepackage{tkz-euclide}\usepackage{tikz}
\usepackage{tikz-cd}
\usetikzlibrary{arrows}
\tikzset{>=stealth,commutative diagrams/.cd,
  arrow style=tikz,diagrams={>=stealth}} %% cool arrow head
\tikzset{shorten <>/.style={ shorten >=#1, shorten <=#1 } } %% allows shorter vectors

\usetikzlibrary{backgrounds} %% for boxes around graphs
\usetikzlibrary{shapes,positioning}  %% Clouds and stars
\usetikzlibrary{matrix} %% for matrix
\usepgfplotslibrary{polar} %% for polar plots
\usepgfplotslibrary{fillbetween} %% to shade area between curves in TikZ
%\usetkzobj{all} %% obsolete

\usepackage[makeroom]{cancel} %% for strike outs
%\usepackage{mathtools} %% for pretty underbrace % Breaks Ximera
%\usepackage{multicol}
\usepackage{pgffor} %% required for integral for loops

\usepackage{tkz-tab}  %% for sign charts

%% http://tex.stackexchange.com/questions/66490/drawing-a-tikz-arc-specifying-the-center
%% Draws beach ball
\tikzset{pics/carc/.style args={#1:#2:#3}{code={\draw[pic actions] (#1:#3) arc(#1:#2:#3);}}}



\usepackage{array}
\setlength{\extrarowheight}{+.1cm}
\newdimen\digitwidth
\settowidth\digitwidth{9}
\def\divrule#1#2{
\noalign{\moveright#1\digitwidth
\vbox{\hrule width#2\digitwidth}}}





\newcommand{\RR}{\mathbb R}
\newcommand{\R}{\mathbb R}
\newcommand{\N}{\mathbb N}
\newcommand{\Z}{\mathbb Z}

\newcommand{\sagemath}{\textsf{SageMath}}


\renewcommand{\d}{\,d}
%\def\d{\mathop{}\!d}
%\def\d{\,d}
\newcommand{\dd}[2][]{\frac{\d #1}{\d #2}}
\newcommand{\pp}[2][]{\frac{\partial #1}{\partial #2}}
\renewcommand{\l}{\ell}
\newcommand{\ddx}{\frac{d}{\d x}}

\newcommand{\zeroOverZero}{\ensuremath{\boldsymbol{\tfrac{0}{0}}}}
\newcommand{\inftyOverInfty}{\ensuremath{\boldsymbol{\tfrac{\infty}{\infty}}}}
\newcommand{\zeroOverInfty}{\ensuremath{\boldsymbol{\tfrac{0}{\infty}}}}
\newcommand{\zeroTimesInfty}{\ensuremath{\small\boldsymbol{0\cdot \infty}}}
\newcommand{\inftyMinusInfty}{\ensuremath{\small\boldsymbol{\infty - \infty}}}
\newcommand{\oneToInfty}{\ensuremath{\boldsymbol{1^\infty}}}
\newcommand{\zeroToZero}{\ensuremath{\boldsymbol{0^0}}}
\newcommand{\inftyToZero}{\ensuremath{\boldsymbol{\infty^0}}}



\newcommand{\numOverZero}{\ensuremath{\boldsymbol{\tfrac{\#}{0}}}}
\newcommand{\dfn}{\textbf}
%\newcommand{\unit}{\,\mathrm}
\newcommand{\unit}{\mathop{}\!\mathrm}
\newcommand{\eval}[1]{\bigg[ #1 \bigg]}
\newcommand{\seq}[1]{\left( #1 \right)}
\renewcommand{\epsilon}{\varepsilon}
\renewcommand{\phi}{\varphi}


\renewcommand{\iff}{\Leftrightarrow}

\DeclareMathOperator{\arccot}{arccot}
\DeclareMathOperator{\arcsec}{arcsec}
\DeclareMathOperator{\arccsc}{arccsc}
\DeclareMathOperator{\si}{Si}
\DeclareMathOperator{\scal}{scal}
\DeclareMathOperator{\sign}{sign}


%% \newcommand{\tightoverset}[2]{% for arrow vec
%%   \mathop{#2}\limits^{\vbox to -.5ex{\kern-0.75ex\hbox{$#1$}\vss}}}
\newcommand{\arrowvec}[1]{{\overset{\rightharpoonup}{#1}}}
%\renewcommand{\vec}[1]{\arrowvec{\mathbf{#1}}}
\renewcommand{\vec}[1]{{\overset{\boldsymbol{\rightharpoonup}}{\mathbf{#1}}}\hspace{0in}}

\newcommand{\point}[1]{\left(#1\right)} %this allows \vector{ to be changed to \vector{ with a quick find and replace
\newcommand{\pt}[1]{\mathbf{#1}} %this allows \vec{ to be changed to \vec{ with a quick find and replace
\newcommand{\Lim}[2]{\lim_{\point{#1} \to \point{#2}}} %Bart, I changed this to point since I want to use it.  It runs through both of the exercise and exerciseE files in limits section, which is why it was in each document to start with.

\DeclareMathOperator{\proj}{\mathbf{proj}}
\newcommand{\veci}{{\boldsymbol{\hat{\imath}}}}
\newcommand{\vecj}{{\boldsymbol{\hat{\jmath}}}}
\newcommand{\veck}{{\boldsymbol{\hat{k}}}}
\newcommand{\vecl}{\vec{\boldsymbol{\l}}}
\newcommand{\uvec}[1]{\mathbf{\hat{#1}}}
\newcommand{\utan}{\mathbf{\hat{t}}}
\newcommand{\unormal}{\mathbf{\hat{n}}}
\newcommand{\ubinormal}{\mathbf{\hat{b}}}

\newcommand{\dotp}{\bullet}
\newcommand{\cross}{\boldsymbol\times}
\newcommand{\grad}{\boldsymbol\nabla}
\newcommand{\divergence}{\grad\dotp}
\newcommand{\curl}{\grad\cross}
%\DeclareMathOperator{\divergence}{divergence}
%\DeclareMathOperator{\curl}[1]{\grad\cross #1}
\newcommand{\lto}{\mathop{\longrightarrow\,}\limits}

\renewcommand{\bar}{\overline}

\colorlet{textColor}{black}
\colorlet{background}{white}
\colorlet{penColor}{blue!50!black} % Color of a curve in a plot
\colorlet{penColor2}{red!50!black}% Color of a curve in a plot
\colorlet{penColor3}{red!50!blue} % Color of a curve in a plot
\colorlet{penColor4}{green!50!black} % Color of a curve in a plot
\colorlet{penColor5}{orange!80!black} % Color of a curve in a plot
\colorlet{penColor6}{yellow!70!black} % Color of a curve in a plot
\colorlet{fill1}{penColor!20} % Color of fill in a plot
\colorlet{fill2}{penColor2!20} % Color of fill in a plot
\colorlet{fillp}{fill1} % Color of positive area
\colorlet{filln}{penColor2!20} % Color of negative area
\colorlet{fill3}{penColor3!20} % Fill
\colorlet{fill4}{penColor4!20} % Fill
\colorlet{fill5}{penColor5!20} % Fill
\colorlet{gridColor}{gray!50} % Color of grid in a plot

\newcommand{\surfaceColor}{violet}
\newcommand{\surfaceColorTwo}{redyellow}
\newcommand{\sliceColor}{greenyellow}




\pgfmathdeclarefunction{gauss}{2}{% gives gaussian
  \pgfmathparse{1/(#2*sqrt(2*pi))*exp(-((x-#1)^2)/(2*#2^2))}%
}


%%%%%%%%%%%%%
%% Vectors
%%%%%%%%%%%%%

%% Simple horiz vectors
\renewcommand{\vector}[1]{\left\langle #1\right\rangle}


%% %% Complex Horiz Vectors with angle brackets
%% \makeatletter
%% \renewcommand{\vector}[2][ , ]{\left\langle%
%%   \def\nextitem{\def\nextitem{#1}}%
%%   \@for \el:=#2\do{\nextitem\el}\right\rangle%
%% }
%% \makeatother

%% %% Vertical Vectors
%% \def\vector#1{\begin{bmatrix}\vecListA#1,,\end{bmatrix}}
%% \def\vecListA#1,{\if,#1,\else #1\cr \expandafter \vecListA \fi}

%%%%%%%%%%%%%
%% End of vectors
%%%%%%%%%%%%%

%\newcommand{\fullwidth}{}
%\newcommand{\normalwidth}{}



%% makes a snazzy t-chart for evaluating functions
%\newenvironment{tchart}{\rowcolors{2}{}{background!90!textColor}\array}{\endarray}

%%This is to help with formatting on future title pages.
\newenvironment{sectionOutcomes}{}{}



%% Flowchart stuff
%\tikzstyle{startstop} = [rectangle, rounded corners, minimum width=3cm, minimum height=1cm,text centered, draw=black]
%\tikzstyle{question} = [rectangle, minimum width=3cm, minimum height=1cm, text centered, draw=black]
%\tikzstyle{decision} = [trapezium, trapezium left angle=70, trapezium right angle=110, minimum width=3cm, minimum height=1cm, text centered, draw=black]
%\tikzstyle{question} = [rectangle, rounded corners, minimum width=3cm, minimum height=1cm,text centered, draw=black]
%\tikzstyle{process} = [rectangle, minimum width=3cm, minimum height=1cm, text centered, draw=black]
%\tikzstyle{decision} = [trapezium, trapezium left angle=70, trapezium right angle=110, minimum width=3cm, minimum height=1cm, text centered, draw=black]


\outcome{Use the first derivative to determine whether a function is increasing or decreasing.}
\outcome{Define higher order derivatives.}
\outcome{Compare differing notations for higher order derivatives.}
\outcome{Identify the relationships between the function and its first and second derivatives.}


\title[Dig-In:]{Higher order derivatives and graphs}

\begin{document}
\begin{abstract}
 Here we make a connection between a graph of a function and its derivative and higher order derivatives.   
\end{abstract}
\maketitle
\begin{definition}\index{increasing}\index{decreasing}
We say that a function $f$ is \textbf{increasing} on an interval $I$ if $f(x_{1})<f(x_{2})$, for all pairs of numbers $x_{1}$, $x_{2}$ in $I$ such that $x_{1}<x_{2}$ .\\
We say that a function $f$ is \textbf{decreasing} on an interval $I$ if $f(x_{1})>f(x_{2})$, for all pairs of numbers $x_{1}$, $x_{2}$ in $I$ such that $x_{1}<x_{2}$ .\\
 \end{definition}

\begin{question}
	\author{Nela Lakos}
	Which of the following famous functions are increasing on $\left(0,\frac{\pi}{2}\right)$?
	\begin{selectAll}
		\choice[correct]{$\sin{(x)}$}
		\choice{$\sin{(2x)}$}
		\choice{$\cos{(x)}$}
		\choice[correct]{$\tan{(x)}$}
		\choice{$\cot{(x)}$}
		\choice[correct]{$f(x)=x^2$}
		\choice{$g(x)=\dfrac{1}{x}$}
	\end{selectAll}
	\begin {explanation} 
		The function $\cos{(x)}$ is not increasing on $I=\Bigl(0,\frac{\pi}{2}\Bigr)$, because  if we take a pair of numbers in $I$, say,  
		$x_{1}=\frac{\pi}{6}$, and   $x_{2}=\frac{\pi}{3}$, then $x_{1}<x_{2}$,  but \\$f(x_{1})>f(x_{2})$,  since 
		$f(x_{1})=\cos{\Bigl(\frac{\pi}{6}\Bigr)}=\frac{\sqrt{3}}{2}$, and $f(x_{2})=\cos{}\Bigl(\frac{\pi}{3}\Bigr)=\frac{1}{2}$.
	 \end{explanation}
\end{question}


\begin{question}
\author{Nela Lakos}
	Consider the graph of the function $f$ below:
	\begin{image}
          \begin{tikzpicture}
	    \begin{axis}[
	        ticks=none,
                domain=-2.5:2.5,
                width=6in,
                height=3in,
                xmin=-1.5, xmax=1.5,
                ymin=-1, ymax=1,
                axis lines =middle, xlabel=$x$, ylabel=$y$,
                every axis y label/.style={at=(current axis.above origin),anchor=south},
                every axis x label/.style={at=(current axis.right of origin),anchor=west},
              ]
              \addplot [very thick, penColor2, smooth, samples=100,domain=-2:2] {x^3-x-1/6};
              \node at (axis cs:1.3,1.5) [penColor2, anchor=west] {$f$};
              \addplot[color=penColor,fill=penColor,only marks,mark=*] coordinates{(-.58,0)};%%closed
              \addplot[color=penColor,fill=penColor,only marks,mark=*] coordinates{(.58,0)};%%closed
              \node at (axis cs:-0.67,-0.1) [penColor, anchor=west] {$a$};
              \node at (axis cs:0.47,-0.1) [penColor, anchor=west] {$b$};
              \node at (axis cs:1.15,0.6) [penColor2] {$f$};
            \end{axis}
          \end{tikzpicture}
        \end{image}
	On which of the following intervals is $f$ increasing?
	\begin{selectAll}
		\choice[correct]{$(-\infty,a)$}
		\choice{$(-\infty,b)$}
		\choice{$(a,b)$}
		\choice{$(a,+\infty)$}
		\choice[correct]{$(b,+\infty)$}
	\end{selectAll}
	\begin {explanation} \
		The function $f$ is not increasing on the interval $(-\infty,b)$, because if we pick a pair of numbers from $(-\infty,b)$, 
		say, $x_{1}=a$, and $x_{2}=0$, then $x_{1}<x_{2}$, but  $f(x_{1})>f(x_{2})$.
	\end {explanation}
\end{question}


Think about the lines tangent to the graph of the function on those intervals you found in this question.  
Is there anything that the slopes of all of those tangent lines have in common?
They are all POSITIVE!  Look in the interval $(a,b)$.  The slopes of tangent lines are all NEGATIVE in that interval.
Notice that the positive slopes occurred when the function was increasing and the negative slopes occurred
when the function was decreasing?  That was no accident.

Since the derivative gives us a formula for the slope of a tangent
line to a curve, we can gain information about a function purely from
the sign of the derivative.  In particular, we have the following theorem
\begin{theorem}[Increasing/Decreasing Test]\index{increasing decreasing test}
	A function $f$ is \textbf{increasing} on any interval $I$ where $f'(x)>0$, for all $x$ in $I$.\\
	A function $f$  is \textbf{decreasing} on any interval $I$ where $f'(x)<0$, for all $x$ in $I$.\\
 \end{theorem}

\begin{question}
  Below we have graphed $y=f(x)$:
  \begin{image}
  \begin{tikzpicture}
	\begin{axis}[
            xmin=-2,xmax=2,ymin=-8,ymax=8,
            axis lines=center,
            width=6in,
            height=3in,
            every axis y label/.style={at=(current axis.above origin),anchor=south},
            every axis x label/.style={at=(current axis.right of origin),anchor=west},
          ]        
          \addplot [very thick,penColor,smooth, domain=(-2:2)] {x^3+x^2-2*x)};
        \end{axis}
  \end{tikzpicture}
  \end{image}
  Is the first derivative positive or negative on the interval $-1<x<1/2$?
  \begin{prompt}
    \begin{multipleChoice}
      \choice{Positive}
      \choice[correct]{Negative}
    \end{multipleChoice}
  \end{prompt}
\end{question}

\begin{question}
  Below we have graphed $y=f'(x)$:
  \begin{image}
  \begin{tikzpicture}
	\begin{axis}[
            xmin=-2,xmax=2,ymin=-8,ymax=8,
            axis lines=center,
            width=6in,
            height=3in,
            every axis y label/.style={at=(current axis.above origin),anchor=south},
            every axis x label/.style={at=(current axis.right of origin),anchor=west},
          ]        
          \addplot [very thick,penColor,smooth, domain=(-2:2)] {x^3+x^2-2*x)};
        \end{axis}
  \end{tikzpicture}
  \end{image}
  Is the function $f$ increasing or decreasing on
  the interval $-1<x<0$?
  \begin{prompt}
    \begin{multipleChoice}
      \choice[correct]{Increasing}
      \choice{Decreasing}
    \end{multipleChoice}
  \end{prompt}
  \begin{explanation} From the graph of  $f'$ we can see that $f'(x)>0$ for all $x$ in $(-1,0)$. Then, the Theorem above implies that the function $f$ is increasing on this interval.
      \end{explanation}
\end{question}

We call the derivative of the derivative the \dfn{second
  derivative}, the derivative of the second derivative (the derivative of the derivative of the derivative) the
\dfn{third derivative}, and so on. We have special notation for
higher derivatives, check it out:
\begin{description}
\item[First derivative:] $\ddx f(x) = f'(x) = f^{(1)}(x)$.
\item[Second derivative:] $\dd[~^2]{x^2} f(x) = f''(x) = f^{(2)}(x)$.
\item[Third derivative:] $\dd[~^3]{x^3} f(x) = f'''(x) = f^{(3)}(x)$.
\end{description}

We use the facts above in our next example.

\begin{example}
  Here we have unlabeled graphs of $f$, $f'$, and $f''$:
  \begin{image}
  \begin{tikzpicture}
	\begin{axis}[
            xmin=-2,xmax=2,ymin=-8,ymax=8,
            axis lines=center,
            ticks=none,
            width=6in,
            height=3in,
            every axis y label/.style={at=(current axis.above origin),anchor=south},
            every axis x label/.style={at=(current axis.right of origin),anchor=west},
          ]        
          \addplot [very thick,penColor,smooth, domain=(-2:2)] {x^3+.3*x^2-2*x)};
          \addlegendentry{$A$};
          \addplot [very thick, dashed,penColor,smooth, domain=(-2:2)] {3*x^2+2*.3*x-2)};
          \addlegendentry{$B$};
          \addplot [very thick, dotted,penColor,smooth, domain=(-2:2)] {6*x+2*.3)};
          \addlegendentry{$C$};
        \end{axis}
  \end{tikzpicture}
  \end{image}
  Identify each curve above as a graph of $f$, $f'$, or $f''$.
  \begin{explanation} 
    Here we see three curves, $A$, $B$, and $C$. Since $A$ is
    \wordChoice{\choice{positive} \choice{negative}
      \choice[correct]{increasing} \choice{decreasing}} when $B$ is
    positive and
    \wordChoice{\choice{positive}\choice{negative}\choice{increasing}\choice[correct]{decreasing}}
    when $B$ is negative, we see
    \[
    A'=B.
    \]
    Since $B$ is increasing when $C$ is
    \wordChoice{\choice[correct]{positive}\choice{negative}\choice{increasing}
      \choice{decreasing}} and decreasing when $C$ is
    \wordChoice{\choice{positive}\choice[correct]{negative}\choice{increasing}\choice{decreasing}}, we see
    \[
    B'=C.
    \]
    Hence $f=A$, $f'=B$, and $f''=C$.
  \end{explanation}
\end{example}




\begin{example}
    Here we have unlabeled graphs of $f$, $f'$, and $f''$:
    \begin{image}
      \begin{tikzpicture}
	\begin{axis}[
            domain=-4:4,
            ticks=none,
            ymax=2, ymin=-2,
            xmax=4, xmin=-4,
            axis lines =middle,
            every axis y label/.style={at=(current axis.above origin),anchor=south},
            every axis x label/.style={at=(current axis.right of origin),anchor=west},
            width=6in,
            height=3in,
          ]
          \addplot [very thick, penColor,smooth,samples=100] {2/(.75*sqrt(2*pi))*exp(-((x)^2)/(2*.75^2)) *(-x)/(.75^2)};
          \addlegendentry{$A$};
          \addplot [very thick, dashed,penColor,smooth,samples=100] {2*gauss(0,.75)};
          \addlegendentry{$B$};
          \addplot [very thick, dotted,penColor,smooth,samples=100] {2/(.75*sqrt(2*pi))*exp(-((x)^2)/(2*.75^2)) *(-1)/(.75^2) +
            2/(.75*sqrt(2*pi))*exp(-((x)^2)/(2*.75^2)) *(x^2)/(.75^4)};
          \addlegendentry{$C$};
        \end{axis}
          \end{tikzpicture}
\end{image}
    Identify each curve above as a graph of $f$, $f'$, or $f''$.
      \begin{explanation} 
        Here we see three curves, $A$, $B$, and $C$. Since $B$ is
        \wordChoice{\choice{positive}\choice{negative}\choice[correct]{increasing}\choice{decreasing}} when $A$
        is positive and
        \wordChoice{\choice{positive}\choice{negative}\choice{increasing}\choice[correct]{decreasing}} when $A$
        is negative, we see
        \[
        B'=A.
        \]
        Since $A$ is increasing when $C$ is
        \wordChoice{\choice[correct]{positive}\choice{negative}\choice{increasing}\choice{decreasing}}
          and decreasing when $C$ is
          \wordChoice{\choice{positive}\choice[correct]{negative}\choice{increasing}\choice{decreasing}}, we
          see
        \[
        A'=C.
        \]
        Hence $f=\answer[given]{B}$, $f'=\answer[given]{A}$, and
        $f''=\answer[given]{C}$.
      \end{explanation}
\end{example}

\begin{example}
  Here we have unlabeled graphs of $f$, $f'$, and $f''$:
  \begin{image}
  \begin{tikzpicture}
	\begin{axis}[
            xmin=-6.75,xmax=6.75,ymin=-1.5,ymax=1.5,
            axis lines=center,
            ticks=none,
            width=6in,
            height=3in,
            every axis y label/.style={at=(current axis.above origin),anchor=south},
            every axis x label/.style={at=(current axis.right of origin),anchor=west},
          ]        
          \addplot [very thick, penColor, samples=100,smooth, domain=(-6.75:6.75)] {-sin(deg(x))};
          \addlegendentry{$A$};
          \addplot [very thick, dashed,penColor, samples=100,smooth, domain=(-6.75:6.75)] {cos(deg(x))};
          \addlegendentry{$B$};
          \addplot [very thick, dotted,penColor, samples=100,smooth, domain=(-6.75:6.75)] {sin(deg(x))};
          \addlegendentry{$C$};
        \end{axis}
  \end{tikzpicture}
  \end{image}
  Identify each curve above as a graph of $f$, $f'$, or $f''$.
  %One is of $f$, another is of $f'$ and a third is of $f''$.  Explain
  %what strategies you could use to identify which graph corresponds
    \begin{explanation} %%BADBAD Need Dropdown
    Here we see three curves, $A$, $B$, and $C$. Since $C$ is
    \wordChoice{\choice{positive}\choice{negative}\choice[correct]{increasing}\choice{decreasing}} when $B$ is
    positive and \wordChoice{\choice{positive}\choice{negative}\choice{increasing}\choice[correct]{decreasing}}
    when $B$ is negative, we see
    \[
    C'=B.
    \]
    Since $B$ is increasing when $A$ is
    \wordChoice{\choice[correct]{positive}\choice{negative}\choice{increasing}\choice{decreasing}} and
    decreasing when $A$ is
    \wordChoice{\choice{positive}\choice[correct]{negative}\choice{increasing}\choice{decreasing}}, we see
    \[
    B'=A.
    \]
    Hence $f=\answer[given]{C}$, $f'=\answer[given]{B}$, and
    $f''=\answer[given]{A}$.
  \end{explanation}
\end{example}


\end{document}
